\documentclass{oblivoir}
\usepackage{graphicx}
\usepackage{mathtools}
\usepackage[left=3.5cm,right=3.5cm,top=3cm,bottom=3cm,a4paper]{geometry}
\usepackage[dvipsnames,svgnames,x11names]{xcolor}
\title{특허 백일장: 환자나 신체 장애자를 위한 의자형 체중계}
\author{1학년 1반 1번 홍길동}

\begin{document}
	\maketitle
	\tableofcontents
	
	이 문서는 \textbf{총 3페이지}로 구성되어 있습니다.
	\section{배경이 되는 기술}
	본 발명은 몸이 불편한 환자나 신체 장애자를 위한 체중계에 관한 것으로 특히, 의자에 앉은 상태에서 체중을 측정할 수 있는 의자형 체중기에 관한 것이 다.
	
	일반적으로 체중을 측정하는 체중 측정기는 판상의 납작한 박스 형태로 이루 어지거나 또는 그외 여러 가지 형태로 이루어지나 이들 모두는 공통적으로, 그 위 에 올라가서 계측된 눈금을 확인하는 것에 의하여 체중을 측정하여 왔었다.
	
	상기한 예 외의 그 외 여러 형태의 체중 측정기도 있으나 이들 모두는 특히 몸이 불편한 사람들이나 신체 장애자들이 사용하기엔 매우 불편한 기구들이었다.
	
	본 발명은 이를 감안하여 발명한 것으로, 본 발명의 목적은 몸이 불편한 사람들이 편리하게 의자 위에 앉은 상태에서 무게를 측정할 수 있게 의자형으로 체중기를 발명한 것으로, 상기의 목적을 달성하기 위한 본 발명은, 의자의 다리부 (1)와 좌석부(50) 사이에 로우드셀(10)을 설치하되, 로우드셀(10)의 하면쪽 일단은 다리부(1)와 체결되고, 로우드셀(10)의 상면쪽 다른 일단은 좌석부(50)와 체결되며, 로우드 셀(10)은 무게를 인식시켜주는 콘트롤판넬(32)과 연결되는 것을 특징으로 한다.
	
	또한, 상기 콘트롤판넬은 무게 표시창이나 음성인식이 가능하게 구성되며, 상기한 좌석부(50)는 좌석(30)과, 그리고 발걸이(22)를 갖는 프레임부(20)로 이루어지며, 프레임부(20) 및 다리부(1)엔 체결부(21,4)를 각각 구성하여, 로우드 셀(10)을 체결하며, 상기 콘트롤판넬(32)은 좌석(30)의 팔걸이(31) 측부에 설치, 측부에서 꺾어 접을 수 있게 설치 또는 좌석(30)의 뒤쪽 중 어느 한 곳에 설치하는 것을 특징으로 한다.
	
	이하, 본 발명의 구성 및 동작을 첨부 도면에 의거 더욱 상세히 설명한다.
	
	\section{문제 해결 방법 (아이디어 구성)}
	도 1 은 본 발명에 대한 개략 분해 사시도이고, 도 2 는 도 1 에 대한 개략 측면도이다 .
	도 1 에서, 의자의 다리부(1)의 상부엔 지지판(2)을 설치하고 그 위엔 로우 드셀(10)이 장착된다. 다리부(1)의 하단 바닥 쪽은 도시한 바와 같이 전방엔 고무체(3)를 설치하여 미끄러지지 못하도록 하고 후방엔 360 회전 가능한 바퀴(13) 를 구성하여 이동이 자유롭게 할 수도 있다.
	지지판(2)엔 로우드셀(10)의 한쪽을 체결할 수 있게 체결공 등과 같은 체결부(4)를 형성시킨다.
	로우드셀(10)의 상부 쪽엔 프레임부(20)를 구성하였다. 
	프레임부(20) 역시 체결공 등으로 구성 가능한 체결부(21)를 형성한다.
	프레임부의 하단쪽엔 발걸이(22)를 구성하였다. 
	프레임부 와 다리부(1) 사이의 로우드셀(10)의 연결은 로우드셀(10)의 하단의 한쪽은 다리부 (1)의 체결부(4)와 견고히 체결하고, 로우드셀(10) 상면의 다른 한쪽은 프레임부 (20)의 체결부(21)와 체결된다.
	
	이를 측면으로 알기 쉽게 도시하면 도 2 와 같은 결합 상태가 된다. 
	
	프레임부(20)의 상부엔 의자의 좌석(30)이 설치된다. 
	이 프레임부(20)와 좌석(30)은 일체로 형성되어 좌석부(50)를 이룬다. 
	프레임부(20)엔 몸이 불편한 사람들을 위하여 팔걸이(31)를 형성하는 것이 바람직하다. 발걸이(31)의 한쪽 측부엔 콘트롤 판넬부(32)를 설치할 수도 있다.
	이 콘트롤판넬부(32)는 팔걸이(31)의 일 측에 지지대와 함께 설치하거나 또는 팔걸이에 팔을 걸칠 때 불편함을 제거하기 위 하여 90도로 아래로 꺾어 접을 수 있게 구성할 수도 있다.
	다른 한편으로는, 콘트롤판넬(32)을 도면에 도시한 바와 같이, 좌석(30)의 머리 뒤쪽 부위 위치에 설 치할 수도 있다. 
	이는 중환자의 경우 본인이 무게를 직접 확인하기 매우 불편한 환자의 경우에 적합하다.
	로우드셀(10)은 콘트롤판넬(32)과 연결되어 있어, 무게가 콘트롤판넬(32)의 표시창에 드러나게 된다. 콘트롤판넬(32)은 예를 들어 전원 온, 오프부, 무게표시창, 제로 세트부 등을 구성할 수 있다. 
	그리고 콘트롤판넬을, 눈으로 확인하기 어려운 환자나 시각장애자를 위하여, 무게 표시창 대신 음성으로 무게를 알려주는 것으로 구성할 수도 있다.
	
	상기한 구성에 의한 동작 관계를 이하, 상세히 설명한다.
	
	먼저 환자나 몸이 불편한 사람의 몸무게를 측정하고자 할 경우, 신체를 좌석 (30)에 앉는다.
	그러면, 그 하중은 프레임부(20)를 거쳐 로우드셀(10)에 전달된다.
	
	
	그러면 콘트롤판넬(32)에 그 측정된 무게가 예로써, LCD 표시창에 나타난 다.
	
	한편, 로우드셀(10)의 상부엔 프레임부(20)와 좌석에 의한 무게를 고려할 수 있으나 이는 이미 로우드셀(10) 설치시 이를 감안하여 "0" 치의 눈금으로 조절되 어 있기 때문에 전혀 문제 될 것이 없는바, 순수히 체중을 측정하고자 하는 자의 무게만 측정된다.
	
	환자일 경우나, 의자에 앉아 있는 것이 곤란한 장애자의 등의 경우엔 발걸이 (22)가 또한 형성되어 있어, 이 발걸이(22) 위에 올라서기만 하여도 이 무게는 로우드셀(10)에 충분히 전달되므로 무게 측정이 용이하게 가능하게 된다.
	
	상기한 본 발명은 한 가지 형태의 의자에 적용한 것으로 한정하여 설명하였지만 본원 발명의 사상은 로우드셀(10)을 의자에 적용하여 장착하는 것인바, 의자의 형태는 어떠한 형태로든 변형이 가능한 것이므로 굳이 본 발명의 설명이 적용된 의자의 형태에 한정되는 것은 결코 아니다.
	
	\section{발명의 효과}
	이상과 같은 본 발명에 의하면, 몸이 불편한 사람이나 환자 또는 신체 장애 자용으로 체중을 측정하기가 매우 편리한 장점이 있다.
	
	\section{도면}
	여기에 도면을 추가하세요.
	
\end{document}