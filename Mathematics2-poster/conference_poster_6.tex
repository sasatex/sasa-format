%%%%%%%%%%%%%%%%%%%%%%%%%%%%%%%%%%%%%%%%%
% a0poster Portrait Poster 
% LaTeX Template
% with University Copenhagen logo
% Version 1.0 (22/06/13)
%
% Based on:
% The a0poster class was created by:
% Gerlinde Kettl and Matthias Weiser (tex@kettl.de)
% 
% This template has been downloaded from:
% http://www.LaTeXTemplates.com11
%
%%%%%%%%%%%%%%%%%%%%%%%%%%%%%%%%%%%%%%%%%

%----------------------------------------------------------------------------------------
%	PACKAGES AND OTHER DOCUMENT CONFIGURATIONS
%----------------------------------------------------------------------------------------

\documentclass[a0,portrait]{a0poster}
\usepackage{kotex}
\usepackage[utf8]{inputenc}

\usepackage{multicol} % This is so we can have multiple columns of text side-by-side
\columnsep=100pt % This is the amount of white space between the columns in the poster
\columnseprule=3pt % This is the thickness of the black line between the columns in the poster

\usepackage[svgnames]{xcolor} % Specify colors by their 'svgnames', for a full list of all colors available see here: http://www.latextemplates.com/svgnames-colors

\usepackage{times} % Use the times font
%\usepackage{palatino} % Uncomment to use the Palatino font

\usepackage{graphicx} % Required for including images
\graphicspath{{figures/}} % Location of the graphics files
\usepackage{booktabs} % Top and bottom rules for table
\usepackage[font=small,labelfont=bf]{caption} % Required for specifying captions to tables and figures
\usepackage{amsfonts, amsmath, amsthm, amssymb} % For math fonts, symbols and environments
\usepackage{wrapfig} % Allows wrapping text around tables and figures
\definecolor{ku}{RGB}{144,26,30}
\definecolor{ku-yellow}{RGB}{255,249,25}\definecolor{kublue}{RGB}{0,25,75}

\usepackage{eso-pic}
               \newcommand\BackgroundIm{
               \put(66,-71){
               \parbox[b][\paperheight]{\paperwidth}{%
               \vfill
               \centering
               \includegraphics[height=\paperheight,width=\paperwidth,
               keepaspectratio]{background.pdf}%
               \vfill
               }}}

\begin{document}
 \AddToShipoutPicture*{\BackgroundIm}

%----------------------------------------------------------------------------------------
%	POSTER HEADER 
%----------------------------------------------------------------------------------------

% The header is divided into two boxes:
% The first is 75% wide and houses the title, subtitle, names, university/organization and contact information
% The second is 25% wide and houses a logo for your university/organization or a photo of you
% The widths of these boxes can be easily edited to accommodate your content as you see fit

\begin{flushright}
\begin{minipage}[b]{0.15\linewidth}
\includegraphics[width=10cm]{emblem.png} \\
\end{minipage}
\end{flushright}

\begin{minipage}[t]{0.60\linewidth}
\Huge \color{NavyBlue} \sffamily{\bfseries 거듭제곱의 합에 대한 일반적 접근 } \color{NavyBlue} % Title
\huge\textit{}\\[1cm] % Subtitle
\color{Black}
\Large \textbf{1208 이승민}\\[0.5cm] % Author(s)
\Large Sejong Academy of Science and Arts\\[0.4cm] % University/organization

\end{minipage}
%
\begin{minipage}[t]{0.40\linewidth}
\flushright
\color{DarkSlateGray}
\Large \textbf{Contact Information:}\\
\textsf{세종특별자치시 아름동}\\
\textsf{달빛1로 265}\\[1cm]
E-mail: \texttt{evander1@sasa.hs.kr}% Email address
\end{minipage}

\vspace{1cm} % A bit of extra whitespace between the header and poster content

%----------------------------------------------------------------------------------------

\begin{multicols}{2} % This is how many columns your poster will be broken into, a portrait poster is generally split into 2 columns

%----------------------------------------------------------------------------------------
%	ABSTRACT
%----------------------------------------------------------------------------------------

\color{ku}% Navy color for the abstract

\begin{abstract} 
\vspace{0.2cm}
\sffamily{거듭제곱의 합에 대한 다양한 접근 방법을 소개한다. 밑이 변하는 경우의 거듭제곱의 합, 멱급수의 합 등에 대해 더욱 일반적인 경우를 계산한다. 멱급수 합의 확장에 대해 탐구한다.}
\end{abstract}


%----------------------------------------------------------------------------------------
%	서론
%----------------------------------------------------------------------------------------

\color{Black} % SaddleBrown color for the introduction

\section*{\sffamily{서론}}
수학 II에서는 급수의 합과 관련하여 특수한 경우들만을 다루었다. 이에 사람들에게 급수의 합의 좀 더 확장된 경우를 알리고자 한다. 이 포스터에서는 거듭제곱의 합에 관하여 수학II에서 다루지 않았던 부분들을 소개하고 탐구할 것이다.

%----------------------------------------------------------------------------------------
%	주요 목표
%----------------------------------------------------------------------------------------

\color{Black}
	
\section*{\sffamily{주요 목표}}

\begin{enumerate}
\item 밑이 변하는 거듭제곱의 합에 대한 일반형태를 소개한다.
\item 멱급수 합의 일반형태를 탐구한다.
\end{enumerate}

%----------------------------------------------------------------------------------------
%	밑이 변하는 거듭제곱의 합
%----------------------------------------------------------------------------------------

\color{DarkSlateGray} % DarkSlateGray color for the rest of the content

\section*{\sffamily{밑이 변하는 거듭제곱의 합}}

가장 기본적인 형태는
\[\sum_{i=0}^{n}i^p \] \\
이다. 수학II에서는 $p=1,2,3$의 경우를 배우는데, 더 높은 차수의 특수한(각 항이 서로 소거되는) 식을 이용하여 유도한다. 이를 일반적 $n$에 대해 생각해보자.

먼저, 식
\[(x+1)^{p+1}-x^{p+1}=\sum_{i=0}^{p}\begin{pmatrix}p+1\\i\end{pmatrix}x^i\] 
가 성립한다. ($\because$ 이항정리) 이 식을 $x=0$부터 $n$까지 더하면, 
\[(n+1)^{p+1}=\sum_{j=0}^n\sum_{i=0}^p\begin{pmatrix}p+1\\i\end{pmatrix}j^i\] 
식에서 좌변은 상수이고 ($\because$소거됨) 우변은 $p$차 이하이다. ($\because$ $i$는 $p$까지 증가) 이때, 우변에서 $\sum_{i=0}^{n}i^p$ 형태의 식을 꺼낼 수 있다.\underline{우변을 차수에 따라 정리한 뒤} 계산하면,
\begin{align*}
\sum_{j=0}^n\sum_{i=0}^p\begin{pmatrix}p+1\\i\end{pmatrix}j^i&=\sum_{i=0}^p\begin{pmatrix}p+1\\i\end{pmatrix}\sum_{j=0}^{n}j^i \\
&=\sum_{i=0}^{p-1}\begin{pmatrix}p+1\\i\end{pmatrix}\sum_{j=0}^{n}j^i+\begin{pmatrix}p+1\\p\end{pmatrix}\sum_{j=0}^{n}j^p 
\end{align*}
식을 정리하면,
\begin{align*}
\begin{pmatrix}p+1\\p\end{pmatrix}\sum_{j=0}^{n}j^p=(n+1)^{p+1}-\sum_{i=0}^{p-1}\begin{pmatrix}p+1\\i\end{pmatrix}\sum_{j=0}^{n}j^i
\end{align*}
따라서,
\begin{equation}
\sum_{i=0}^{n}i^p=\frac{(n+1)^{p+1}}{p+1}-\frac{1}{p+1}\sum_{j=0}^{p-1}\begin{pmatrix}p+1\\j\end{pmatrix}\sum_{i=0}^{n}i^j
\label{1}
\end{equation}

\vspace{1cm}
이로서 $0\thicksim n$까지의 $p$제곱들의 합을 귀납적으로 나타낸 식 \ref{1}을 구했다. 이를 통해 $p=1,2,3$만이 아닌 더 큰 $p$에 대해서도 거듭제곱의 합을 계산할 수 있을 것이다.

\begin{align*}
\sum_{k=0}^n k^2&=\frac 1 6 n(n+1)(2n+1)\\
\sum_{k=0}^n k^3&=\frac 1 4 n^2 (n+1)^2 \\
\sum_{k=0}^n k^4&=\frac 1 {30} n (n+1)(2n+1)(3n^2+3n-1)\\
\sum_{k=0}^n k^5&=\frac 1 {12} n^2 (n+1)^2 (2n^2 + 2n-1) \\
\sum_{k=0}^n k^6&=\frac 1 {42} n(n+1)(2n+1)(3n^4 + 6n^3 - 3n + 1)\\
\sum_{k=0}^n k^7&=\frac 1 {24} n^2 (n+1)^2 (3n^4 + 6n^3 - n^2 - 4n + 2)\\
\sum_{k=0}^n k^8&=\frac 1 {90} n(n+1)(2n+1)(5n^6 + 15n^5 + 5n^4 - 15n^3 - n^2 + 9n - 3)\\
\sum_{k=0}^n k^9&=\frac 1 {20} n^2(n+1)^2 (n^2+n-1)(2n^4 + 4n^3 - n^2 - 3n + 3)\\
\end{align*}




%----------------------------------------------------------------------------------------
%	멱급수의 합
%----------------------------------------------------------------------------------------

\section*{\sffamily{멱급수의 합}}

수학II에서, 멱급수의 간단한 형태는 다음과 같이 주어진다.
\[\sum_{i=1}^{n}ix^i\] 
수학 II에서는 이를 $S$로 치환하여 적절히 계산하였다.
\[S=\sum_{i=1}^{n}ix^i=x+2x^2+3x^3 + \cdots +(n-1)x^{n-1}+nx^n\]
양변에 $x$를 곱한 후 원래 식에서 빼면 중간부분의 항들을 등비수열의 합 형태로 바꿀 수 있다.
\begin{align*}
S&=x+2x^2+3x^3+4x^4+\cdots +(n-1)x^{n-1}+nx^n \\
xS&=\,\qquad x^2+2x^3+3x^4+\cdots +(n-2)x^{n-1}+(n-1)x^{n}+nx^{n+1} 
\end{align*}
\begin{align*}
S&=\frac{1}{1-x}\left(x+x^2+x^3+\cdots+x^{n-1}+x^n-nx^{n+1} \right) \\
&=\frac{1}{1-x}\left(x\frac{1-x^{n}}{1-x}-nx^{n+1} \right)
\end{align*}
그러나, 이를 보다 간편하게 할 수 있는 방법이 있다.
\[\sum_{i=1}^{n}x^i=\frac{x^{n+1}-1}{x-1}\]
식의 양변을 미분하고 $x$를 곱하자.
\[\sum_{i=1}^{n}ix^i=\frac{x-(n+1)x^{n+1}+nx^{n+2}}{(x-1)^2}\]
계산해보면 위 식과 같음을 알 수 있다. \\

이것은 $\sum_{i=1}^{n}ix^i$에서 $\sum_{i=1}^{n}i^kx^i$로 멱급수를 확장할 때 좀 더 간단한 표현이 가능하도록 해준다. $x$에 관한 식 $g(x)$에 대해 $f(x)$를 다음과 같이 정의하자.
\[f\left(g(x)\right)=x\times g^\prime(x)\]
따라서,
\begin{align*}
f\left(\sum_{i=1}^{n}i^{k-1}x^i\right)&=x\times\left(\sum_{i=1}^{n}i^{k-1}x^i\right)^\prime=\sum_{i=1}^{n}i^kx^i
\end{align*}
그러므로,
\begin{align}
\sum_{i=1}^{n}i^kx^i=f\left(\sum_{i=1}^{n}i^{k-1}x^i\right)&=f^2\left(\sum_{i=1}^{n}i^{k-2}x^i\right) \nonumber \\
&=f^{k}\left(\sum_{i=1}^{n}x^i\right)=f^{k}\left(\frac{x^{n+1}-1}{x-1}\right)
\label{2}
\end{align} \\

이로서 보다 일반적 형태의 멱급수의 합을 함수 $f$로 나타낸 식 \ref{2}를 구했다. 기존의 방법대로 계산했다면 $x$를 곱하고 빼는 과정을 여러 번 반복해야 했을 테니 시간이 오래 걸렸을 것이다. 식 \ref{2}를 통해 멱급수 합의 일반식을 더 빠르게 구할 수 있을 것이라 생각된다.
%----------------------------------------------------------------------------------------
%	CONCLUSIONS
%----------------------------------------------------------------------------------------

\color{SaddleBrown} % SaddleBrown color for the conclusions to make them stand out

\section*{\sffamily{결론 및 제언}}

\begin{itemize}
\item 식 \ref{1}은 $p$제곱들의 합을  $p-1$이하의 제곱들의 합으로 나타낸 식이다. 따라서 $p$제곱들의 합을 구하기 위해서는 그 이하 차수의 합공식들이 필요하다. 베르누이 수와 오일러-매클로린 공식을 활용하면 식 \ref{1}을 귀납적이지 않은 식으로 바꿀 수 있지만, 이는 수학을 좀 더 공부한 뒤 알아보아야 할 것 같다.
\item 식 \ref{2}또한 $x$로 표현될 실제 식을 구하기 위해서는 $f^1\thicksim f^{k-1}$까지의 값을 모두 구해야 한다. 만일 실제 식을 $x$와 $k$로 나타낼 수 있다면 계산이 더욱 간편해질 것이라 생각된다.
\end{itemize}

\color{DarkSlateGray} % Set the color back to DarkSlateGray for the rest of the content

%----------------------------------------------------------------------------------------
%	FORTHCOMING RESEARCH
%----------------------------------------------------------------------------------------


 %----------------------------------------------------------------------------------------
%	REFERENCES
%----------------------------------------------------------------------------------------


%----------------------------------------------------------------------------------------
%	ACKNOWLEDGEMENTS
%----------------------------------------------------------------------------------------


%----------------------------------------------------------------------------------------

\begin{minipage}[b]{0.6\linewidth}
{\ } \phantom{}
%includegraphics[width=20cm]{logo.png}\\
\end{minipage}

\begin{flushright}
\begin{minipage}[b]{0.6\linewidth}
\includegraphics[width=20cm]{logo6-1.png}\\
\end{minipage}
\end{flushright}

\end{multicols}

\end{document}